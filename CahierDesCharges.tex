\documentclass[12pt,a4paper]{article}

\usepackage[utf8]{inputenc}
\usepackage[T1]{fontenc}
\usepackage[french]{babel}
\usepackage{fancyhdr}
\usepackage{graphicx}
\usepackage{tikz}
\usepackage{colortbl}
\fancyhead{}
\fancyhead[R]{\slshape \rightmark}
\fancyfoot[R]{\includegraphics[scale=0.05]{logo.png}}
\fancyfoot[L]{\includegraphics[scale=0.015]{logo-epita.png}}

\begin{document}
\pagestyle{fancy}
\thispagestyle{empty}
\baselineskip = 20 pt
\newpage
\vspace*{\stretch{1}}

\begin{center}
\begin{Large}
\textbf{Introduction}
\end{Large}
\end{center}

  	\paragraph{		 Le rapport de soutenance qui suit a pour but de présenter le projet du groupe
Error 523.\newline} 

	Nous allons vous présenter les premiers avancements de notre projet Origin Is Unreachable témoignant de la bonne cohésion des membres du groupe ainsi que du bon déroulement de ce projet. Chaque membre a trouvé sa place au sein du groupe et permet une avancée rapide et efficace. Notre projet avance relativement vite tout en respectant les règles fixées par le cahier des charges. 
\vspace*{\stretch{1}}
\newpage
\thispagestyle{empty}
\tableofcontents

\newpage
\section {Rappels}
\subsection{Répartition des tâches}
\paragraph{}
\begin{center}
\begin{tabular}{|c|c|c|c|}
\hline 
\rowcolor{cyan} Avancement & Soutenance 1 & Soutenance 2 & Soutenance 3 \\ 
\hline 
\cellcolor{lightgray} &\multicolumn{3}{|c|}{\cellcolor{lightgray} Solo } \\
\hline 
\cellcolor{lightgray}Graphisme & \cellcolor{yellow} 10\% & \cellcolor{yellow} 30\% & \cellcolor{orange} 80\%\\ 
\hline 
\cellcolor{lightgray}Script & \cellcolor{yellow} 30\% &  \cellcolor{orange} 60\% & \cellcolor{red} 95\% \\ 
\hline 
\cellcolor{lightgray}Scénario & \cellcolor{orange} 50\%&  \cellcolor{orange}80\% &  \cellcolor{red}100\% \\ 
\hline 
\cellcolor{lightgray}Site Web & \cellcolor{yellow} 10\% &  \cellcolor{orange}50\% & \cellcolor{red} 90\% \\ 
\hline
\cellcolor{lightgray}Musique/Son & 0\% & \cellcolor{yellow} 20\% &  \cellcolor{orange} 50\% \\ 
\hline 
\cellcolor{lightgray} &\multicolumn{3}{|c|}{\cellcolor{lightgray} Multi-Joueur } \\
\hline 
\cellcolor{lightgray} Client & \cellcolor{yellow} 5\% & \cellcolor{orange} 30\% & \cellcolor{red} 90\% \\ 
\hline 
\cellcolor{lightgray} Serveur & \cellcolor{yellow} 10\% & \cellcolor{orange} 40\% & \cellcolor{red} 90\% \\ 
\hline 

\end{tabular}
\paragraph{}
\begin{tabular}{|c|c|c|c|c|}
 \hline 
 \rowcolor{cyan}Tâches \ Pers. & Trasher & Tibal & Zaphod & Flow \\ 
 \hline 
 
 \cellcolor{lightgray}Graphisme & \cellcolor{red} &  &  & \cellcolor{yellow} \\ 
 \hline 
 \cellcolor{lightgray}Script &  & \cellcolor{yellow} & \cellcolor{red} & \cellcolor{yellow} \\ 
 \hline 
  \cellcolor{lightgray} Interface Utilisateur & \cellcolor{yellow}  &  &  & \cellcolor{orange} \\ \cellcolor{lightgray} et Monstre & \cellcolor{yellow}  &  &  & \cellcolor{orange} \\ 
 \hline 
  \cellcolor{lightgray}I.A. &  &  & \cellcolor{red} &  \\ 
 \hline 
  \cellcolor{lightgray}Énigmes & \cellcolor{orange} & \cellcolor{orange} &  &  \\ 
 \hline 
 \cellcolor{lightgray}Scénario & \cellcolor{yellow} & \cellcolor{yellow} & \cellcolor{yellow} & \cellcolor{yellow} \\ 
 \hline 
 \cellcolor{lightgray}Site Web &  & \cellcolor{red} &  &  \\ 
 \hline 
 \cellcolor{lightgray} Réseau & \cellcolor{orange} &  & \cellcolor{yellow}  & \cellcolor{red}  \\ 
 \hline 
 \cellcolor{lightgray}Musique/Son &  & \cellcolor{yellow} &  & \cellcolor{yellow} \\ 
 \hline 
 \end{tabular}
\paragraph{Plus on va vers le rouge plus le niveau d'implication dans le domaine est élevé.} 
\paragraph{}
 \begin{tabular}{|c|c|c|c|}
 \hline 
\cellcolor{white} & \cellcolor{yellow} & \cellcolor{orange} & \cellcolor{red} \\ 
 \hline 
 \end{tabular}

\end{center}
\newpage
\subsection{Présentation des membres}
\baselineskip = 16 pt
\begin{center}
\textbf{Thomas VATON "Trasher" - 18 ans - Ancien Terminale S\newline
Game designer et Programmeur réseaux}
\end{center}

Intéressé par le milieu de l'informatique et ayant été souvent entouré par des personnes gravitant autour de ce domaine, cela m'a rapidement passionné. N'ayant néanmoins que très peu pratiqué avant mon entrée à EPITA, ce fut une grande découverte, qui a été et qui est toujours, à la hauteur de mes espérances.

Au sein du groupe Error 523, je suis principalement chargé du game design mais aussi de la partie réseaux et multi-joueurs, ce qui va consister en une tâche de très grande importance, d'autant plus que j'aimerais faire du réseau dans la suite de mon cursus à EPITA.
   
J'espère, grâce à ce projet, pouvoir améliorer mes compétences dans le monde de l'informatique, mais aussi dans ma capacité à travailler en groupe, et je compte bien mener ce projet à son terme avec le reste de mon équipe.

\begin{center}
\textbf{Thibault LEMPEREUR "Tibal" - 18 ans - Ancien Terminal S\newline Développeur web et Programmeur scripts}

\end{center}

J'ai intégré l'EPITA car je suis passionné par l'informatique depuis maintenant plusieurs années. J'étais néanmoins débutant en programmation mais j'ai redoublé d'efforts pour rattraper mon retard. Cependant je n ai pas encore eu l'occasion de participer à un projet comme celui-ci, ce qui est donc une très bonne opportunité. En effet ce projet va me permettre de gagner en maturité mais également de pouvoir progresser aussi bien en programmation qu'en travail d'équipe. Le projet Origin is Unreachable est un projet qui me tient à coeur et je vais m'investir au maximum dans ce dernier. Mon rôle au sein de ce groupe sera le développement du site web, le scénario, le script et la partie audio. 
\newpage
\begin{center}
\textbf{Maxime CHARDON « Zaphod» - 18 ans – Ancien Terminal S
Programmeur scripts et Programmeur réseaux}
\end{center}

Ayant un entourage travaillant dans le milieu de l’informatique et de par l’ampleur que prend celle-ci dans la société autant actuelle que future, l’idée de travailler dans ce domaine m'est apparue nettement.

Malgré cela je n’ai que très peu pratiqué avant mon entré à l’EPITA et ce premier semestre était à la hauteur de mes espérances. J’ai très vite voulu participer à des projets de groupe ou individuel et en ayant pris connaissance du projet de fin d’année, j’ai vite commencé à étudier la programmation avec Unity. J’ai aussi pris part à de nombreuses conférences ou évènements comme une « Jam » où nous avons avec un camarade rendu un jeu fait en une dizaine d’heures avec l’aide de Unity. Voilà pourquoi au sein de ce projet je m’occuperai principalement des scripts CSharp avec Unity mais aussi de la partie réseau.

J’espère de ce projet qu’il m’apporte une certaine expérience dans le domaine de l’informatique, mais aussi dans le travail d’équipe qui est selon moi quelque chose de primordial dans ce milieu. Je compte mener ce projet à son terme avec mon équipe.

\begin{center}
\textbf{Florentin SOUBRAND « Flow» - 18 ans – Ancien Terminal S \newline
Game desin, Programmeur réseaux et Chef de Projet}
\end{center}
	Je me suis tourné vers l'informatique depuis l'âge de 14 ans et ai exploré quelques langages de programmation jusqu'au début du lycée. C'est pourquoi mon orientation ne posait pas de problème. Ainsi mes débuts à EPITA furent assez simples du côté de l'algorithmique et de la programmation.

	Mais à présent tout est nouveau, je ne connaissais ni Unity ni réellement le CSharp, il y encore 2 semaines. J'attends de ce projet qu'il m'aide à partager mes connaissances et en acquérir de nouvelles grâce au groupe. De plus étant chef du groupe je vais devoir apprendre à gérer une équipe ainsi que les conflits qui y surviendront.Ce qui me parait être une très bonne expérience pour la suite de ma scolarité.
	
	Le cursus et les projets proposés sont engageants et encourageants : ils m'ont permis de m'investir pleinement dans les TP et dans ce Jeu que nous allons réaliser durant cette fin d'année.
\baselineskip = 20 pt
\newpage
\section{Notre Projet aujourd'hui}
\subsection{Origine du projet}
L'idée du projet nous est venue lors d'une discussion autour des films
d'horreur sur un trajet de retour de l'école. On s'est alors demandé comment rendre ce
jeu intéressant et différent des autres tout en restant de près ou de loin accessible à nos capacités.
La suite nous est venue naturellement en développant
notre idée en groupe. Étant partie d'une simple idée de jeu d'horreur, cette dernière a progressivement germé et
nous avons fini par décider de faire évoluer le personnage dans une pyramide, changeant un peu du
décor d'une simple maison ou d'une ferme, que l'on retrouve fréquemment dans les jeux de ce style.
Nous avons pu puiser nos idées de différents autres jeux, tel Slender, Resident Evil ou encore Outlast,
pour pouvoir imaginer le nôtre. Malgré un décor complètement différent de ces derniers, nous
comptons bien rendre le jeu un tant soit peu horrifique du moins dans l'ambiance.
Des courses dans des labyrinthes et des énigmes à résoudre vous attendent donc dans la suite de cette
description...

\newpage
\subsection{Gameplay}
Pour la partie solo le joueur est dans une pyramide, dont il doit s'échapper, coincé par une momie qui le pourchasse. Pour finir le jeu, ce dernier devra retrouver des fragments de clé qui lui rappelleront son chemin jusqu'ici tout en lui permettant de sortir de sa terrible situation. Son parcours étant alterné de salles à énigmes et de courses poursuite dans un labyrinthe.
Pour la partie multi-joueur, il se déroule à 5 maximum, avec un joueur qui incarnera la momie et pourra à sa guise placer des pièges ou déchiqueter ses victimes.
Les autres devront comme dans la partie solo essayer de s'échapper.
\paragraph{}
\includegraphics[scale=0.2]{firstRoom.png}
\newpage
\subsubsection{In Game}
Un homme se retrouve seul, dans une pièce sombre, sans aucun souvenir...
Très vite, il se rend compte que cette pièce, dont les murs sont couverts de hiéroglyphes et dont le sol est jonché d'os, qui doivent être là depuis plusieurs centaine d'années, n'est pas un endroit banal....
Mais il a beau chercher, aucun souvenir ne lui revient....
Il décide alors de tenter de s'échapper....
Sortir de la salle ? Il y arrivera sûrement... Mais échapper à la momie qui le pourchassera dans le labyrinthe et les salles suivantes ? Tout devient incertain...
Si néanmoins il y parvient, il trouvera sur son chemin les fragments d'une mystérieuse clé, qui en plus de lui ouvrir la porte de la sortie finale, lui ouvrira certainement les portes de sa mémoire...
\paragraph{}
Pour imager tout ça nous avons dû commencer par créer une map :
\begin{center}
\includegraphics[scale=0.2]{vueMaze3D.png}
\end{center}

Nous avons commencer par la première salle dans laquelle les murs ont été implémenter ainsi que lez sol et le plafond a partir de modèles 3D Unity puis pour étoffer le tout nous avons rajouté des colonnes et des statues è partir de modèles récupérer sur www.blendswap.com, nous avons pris de plus sur ce site les différentes matières et textures. Puis est venue la création des labyrinthe que nous avons décidé de générer procéduralement. ayant commencer par les faire a la main, nous nous sommes vite rendu compte, une fois la génération procédural disponible, que cette dernière rendait beaucoup mieux et nous avons décidé de tout remplacer. Encore une fois nous avons utiliser les textures prisent sur le même site.
Puis il a fallu rendre l'ambiance oppressante: pour se faire, régler l'ambiance lumineuse fut la première chose a faire. Nous avons alors plongé la scène dans le noir et rajouté des torches pour améliorer la luminosité.
Enfin nous avons du implémenter un joueur, un ennemi ainsi que leurs contrôles :
\begin{itemize}
\item[-] le joueur : nous avons écrit un script afin de le contrôler en vue a la première personne et diriger sa vue a la souris et nous avons pris l'asset graphique sur le même site
\item[-] la momie : nous lui avons implémenter une I.A. basique qui sera étoffée par la suite : elle poursuit le joueur dans le labyrinthe et l'attaque une fois assez proche
\end{itemize} 

\newpage
\subsubsection{Énigme}
L'implémentation de la première énigme consiste en un problème mathématique et d'un concept original du résultat sous forme binaire.
\begin{center}
\includegraphics[scale=0.2]{imageEnigme.png}
\end{center}

\subsubsection{Génération procédurale de map}
Nous nous sommes attelés à coder un script pour faire une génération procédurale de labyrinthe, ce script après plusieurs heures de travail et plusieurs reprises du code, nous permet de générer aléatoirement des labyrinthes en ayant pour seuls paramètres les dimensions que nous voulons lui appliquer.


\newpage
\subsubsection{Menus}
L'implémentation des menus s'est faite sans trop de problème, nous avons créé 4 rubriques:


\paragraph{}
\begin{center}
\includegraphics[scale=0.2]{MainMenu.png}
\end{center}

\item[-] PLAY: qui permet de lancer une partie, pour le moment il lance automatiquement la partie car nous n'avons qu'une seule carte (les suivantes viendront plus tard), par la suite ce menu permettra de choisir entre différentes cartes, niveaux...
Mais aussi sûrement de pouvoir choisir son personnage, le monstre qui sera à sa poursuite et différents autres possibilités de modifications.

\newpage

\paragraph{}
\begin{center}
\includegraphics[scale=0.15]{MultiMenu.png}
\end{center}

\item[-] MULTI-PLAYER: qui permettra par la suite de pouvoir lancer une partie à plusieurs, pour l'instant le multijoueur ne fonctionnant pas, ce menu ne change pas grand chose mais aura une grande utilité par la suite. Les parties pourront se dérouler jusqu'à 5 avec plusieurs personnes tentant de s'échapper et un autre joueur pour incarner la momie qui sera à leurs poursuite.

\paragraph{}
\begin{center}
\includegraphics[scale=0.15]{PauseMenu.png}
\end{center}

\item[-] PAUSE: Le menu de pause du jeu, il permet d'accéder aux autres différents menu décrit ci-dessus et dans la rubrique suivante.

\paragraph{}
\begin{center}
\includegraphics[scale=0.15]{OptionsMenu.png}
\end{center}

\item[-] OPTION: il permet comme son nom le dit, de modifier les différents paramètres du jeu tel que le son, celui des musiques, des effets etc... Il peut s'avérer pratique, notamment si votre machine refuse catégoriquement que vous quittiez ce jeu formidable pour tenter de faire autre chose que d'y jouer, mais que le son est vraiment trop fort et que vous êtes terrorisé par l'ambiance si attirante mais à la fois terriblement effrayante de ce jeu d'horreur...

\item[-] QUIT!: et enfin cette dernière, rubrique la plus inutile, permettant de quitter le jeu, car de toute façons vous ne voudrez jamais le quitter car il est vraiment cool.

\newpage

\subsection{Intérêts individuels}
\paragraph{}
Gagner en autonomie, apprendre à utiliser de nouveaux logiciels (tel Unity, TeXmaker ou encore GitHub), mettre en pratique les cours d'algorithmique et de programmation ou encore approfondir nos connaissances dans des domaines pas forcément abordés en cours, sont des avantages que ce projet nous apporte à tous.
De plus les soutenances vont nous permettre de travailler nos qualités à l'oral, que ce soit l'élocution ou encore la gestion du stress. Ce projet, étant jusqu'à maintenant réalisé sérieusement ne peut être que bénéfique dans les apports qu'il peut nous procurer.
Cette première partie de projet nous a donné une vision global du travail qu'il nous reste à accomplir afin d'obtenir un jeu suffisamment complet afin de touché un plus grand publique. Mais aussi de nombreuses notions qui nous étaient encore jusque là floue nous apparaissent aujourd'hui un peu plus familière tel que le réseau (même si le réseau sous Unity est grandement facilité) ou encore l'importance des méthodes du CSharp.

\newpage

\section{Ressources utilisées}
\subsection{Le matériel}
Le matériel principal est constitué de nos ordinateurs, néanmoins nous avons aussi utilisé une tablette graphique
pour la création des logos.
La majeure partie du travail se fera donc à partir des ordinateurs, même si la consommation de nourriture
peu recommandée pour le régime est nécessaire pour travailler dans les conditions les plus correctes ! (On notera par ailleurs le budget café, kebab, pizzas et bières plus qu'important)

\subsection{Ressources Économiques}
Les ressources économiques se résument très rapidement, possédant déjà l'ensemble des logiciels et ressources nécessaires, néanmoins, comme dit précédemment le trou présent dans notre budget est essentiellement du à l'achat de café, bière, pizzas et kebab en tout genre...

\newpage

\subsection{Les logiciels}
Nous utilisons les logiciels suivants pour mener notre projet à bien:
\begin{itemize}
\item[-] Unity : moteur de jeu 2D/3D permettant notamment à l'aide d'outils
intuitifs (moteur physique, audio...) de réaliser des jeux en réseau et
des animations.
\item[-] Photoshop CC : retouche, traitement et dessin par
ordinateur qui servira à la réalisation des graphismes 2D.
\item[-] GIT : logiciel de gestion de versions qui permettra au groupe d'échanger du code et d’autres informations.
\item[-] After effect / Sony Vegas Pro : permettant de réaliser des animations.
\item[-] NotePad ++ : éditeur de texte générique qui servira à la réalisation
du site internet
\item[-] MAGIX Music Maker / FL Studio : utiles pour la
réalisation des bandes son du jeu
\item[-] Visual Studio 2017 / JetBrain's Rider : permettront la programmation en CSharp
\item[-] Blender: permettront de réaliser les objets 3D présents dans le jeu
\end{itemize}

\newpage
\section{Découpage du projet}
\subsection{Découpage du jeu: description des différentes \newline parties}
Découpage du projet:
\begin{itemize}
\item[-] Écriture du scénario et des dialogues.
\item[-] Graphisme: tout ce qui est design des objets, personnages, environnements.
\item[-] Réseaux: la création d'un réseau permettant notamment l'implémentation de la partie multi-joueurs.
\item[-] Gestion des événements, interfaces, et utilisateurs.
\item[-] Implémentation d'une Intelligence Artificielle et d'une génération procédurale de labyrinthe.
\item[-] Création et gestion du site internet.
\item[-] Implémentation des musiques, sons et bruitages.
\item[-] Rédaction en LaTeX du rapport de soutenance.
\end{itemize}

\newpage
\baselineskip = 18 pt
\section{Conclusion}
Ce projet est donc un défi que nous relevons du mieux que nous pouvons, nous mettons le meilleur de nous-même pour le réussir le mieux possible.

Il représente un enjeu de taille du fait que nous débutons globalement dans le milieu de l'informatique, il permet de développer nos compétences en même temps que la réalisation d'un projet plus que concret : aucun d'entre nous ne se serait imaginé, il y a ne serait-ce que deux ans, qu'un jour il créerait un jeu vidéo !.

Nous nous efforçons de coller au cahier des charges (et pour l'instant cela se déroule bien) qui nous permet notamment de planifier notre travail, à la fin de ce projet nous serons donc en mesure de proposer un jeu vidéo en 3D assez complet dans le style que nous avons décidé de lui donner.

En somme, ce projet nécessite un travail de longue haleine qui permet aux élèves du groupe Error 523 de mettre en pratique tout ce qu'ils ont appris et tout ce qu'ils apprendront par eux-mêmes.

Le projet est réalisé correctement jusqu'à maintenant, et le rendu du jeu commence à devenir intéressant, nous nous prenons même à notre propre jeu et nous amusons à jouer sur notre propre jeu.
C'est un travail intéressant qui nous permet effectivement de mettre en pratique ce que nous apprenons en cour de programmation et d'algorithmique, nous allons donc continuer à, jusqu'au dernier moment, développer et faire évoluer notre jeu pour quùil soit toujours plus attractif et fun à essayer ! 
\end{document}